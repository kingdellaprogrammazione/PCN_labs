\subsubsection{Doppler and Sub-Doppler Spectroscopy}

The spectroscopic measurements were performed by sweeping the laser frequency across the Cesium D1 transition. The recorded probe transmission signal reveals the convolution of two physical effects: the broad absorption profile caused by the thermal motion of the atoms (Doppler broadening), and the narrow sub-Doppler resonances arising from the saturation of zero-velocity atoms by the counter-propagating pump beam.

\paragraph{Interpretation of Hyperfine Transitions}
The collected spectrum exhibits four resolved resonance peaks. These features correspond to the allowed optical transitions between the hyperfine sublevels of the ground state ($6^2S_{1/2}, F_g=3, 4$) and the excited state ($6^2P_{1/2}, F_e=3, 4$). The spectrum is divided into two widely separated groups due to the large splitting of the ground state ($\approx \SI{9.193}{\giga\hertz}$). Inside each group, the smaller splitting of the excited state ($\approx \SI{1.168}{\giga\hertz}$) makes the absorption appear as a pair of closely spaced peaks.

\begin{figure}[ht]
    \centering
    \includegraphics[width=0.6\textwidth]{Cs_133_D1_Hyperfine_splitting.png}
    \caption{Energy level diagram of the Cesium D1 transition showing the relevant hyperfine levels \cite{steck}.}
    \label{fig:energy_levels}
\end{figure}



\paragraph{Frequency Calibration}
While the Michelson interferometer provided a relative linearization of the frequency axis, absolute frequency calibration was achieved using the sub-Doppler resonance peaks. We identified the resonance centers of the hyperfine transitions within the sub-Doppler profile. Using the known hyperfine splitting values for the $^{133}\text{Cs}$ D1 line (ground state hyperfine splitting $\Delta \nu_{hfs} = \SI{9.192631770}{\giga\hertz}$ \cite{steck}), we assigned absolute frequency detunings to our linearized axis, achieving a fine-tuning of the frequency scale.