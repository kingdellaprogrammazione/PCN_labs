The experimental setup implements a pump-probe saturation spectroscopy scheme to investigate the hyperfine structure of the Cesium D1 transition. The optical layout is illustrated in Figure \ref{fig:setup}.



The light source is a Distributed Bragg Reflector (DBR) laser diode emitting at approximately \SI{894}{\nano\meter}. To control the intensity in each branch of the setup, the optical power is regulated by rotating half-wave plates ($\lambda/2$) placed immediately upstream of the polarizing beam splitters. This configuration allows for variable splitting ratios before the beam is divided into the different paths:

\begin{enumerate}
    \item \textbf{Interferometry and Calibration:} A portion of the beam is directed into a Michelson interferometer with a fixed path length difference of $\Delta L = \SI{18.7}{\centi\meter}$. The interference fringes produced by this arm are recorded to monitor the relative frequency change of the laser and are used to linearize the frequency axis during data analysis.
    
    \item \textbf{Absorption Probe Path:} A weak portion of the beam, designated as the ``probe'', is directed through the Cesium vapor cell. After interacting with the atomic vapor, its transmitted intensity is detected by a photodiode. This signal constitutes the primary measurement of the atomic absorption spectrum.

    \item \textbf{Pumping Laser Path:} The stronger portion of the beam serves as the ``pump''. It is routed to propagate through the vapor cell in the counter-propagating direction. The pump beam is aligned to spatially overlap with the probe beam within the cell; this saturates the atomic transitions for atoms with zero longitudinal velocity, enabling the observation of sub-Doppler peaks.
\end{enumerate}

The signals detected by the photodiodes are acquired on an oscilloscope, together with the triangular wave used to drive the laser frequency.

\begin{figure}[ht]
    \centering
    \includegraphics[width=0.75\textwidth]{spectroscopy/images/spectroscopy_setup.png}
    \caption{Experimental setup for Saturated Absorption Spectroscopy including the Michelson interferometer for frequency calibration.}
    \label{fig:setup}
\end{figure}


