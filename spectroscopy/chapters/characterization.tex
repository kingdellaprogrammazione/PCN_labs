

The goal of this section is to determine the threshold current ($I_{th}$) of the DBR laser diode, characterizing its transition from spontaneous emission (LED mode) to stimulated emission (lasing mode).

\subsection{Laser Current/Power Characteristic}

We measured the optical output power of the laser as a function of the injection current. The current was increased in steps, and the corresponding power was recorded using a standard optical power meter.



\begin{figure}[ht]
    \centering
    \begin{minipage}{0.48\textwidth}
        \centering
        \includegraphics[width=\linewidth]{spectroscopy/images/laser_power-current.png}
        \caption{Output optical power versus injection current for the DBR laser diode.}
        \label{fig:PI_laser}
    \end{minipage}
    \hfill
    \begin{minipage}{0.48\textwidth}
        \centering
        \includegraphics[width=\linewidth]{spectroscopy/images/laser_power-current_zoom.png}
        \caption{Zoom on the threshold zone.}
    \end{minipage}
\end{figure}

\subsubsection*{Results and Analysis}
The $P-I$ curve (Figure \ref{fig:PI_laser}) exhibits a characteristic "knee" behavior. Below the threshold, the output power is negligible, governed by spontaneous emission. Above the threshold, the power increases linearly with the current due to stimulated emission.

We determined the threshold current to be:
\begin{equation}
    I_{th} \approx \SI{56}{\milli\ampere}
\end{equation}
