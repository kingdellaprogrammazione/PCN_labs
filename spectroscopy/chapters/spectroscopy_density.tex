\subsubsection{Calculation of Atomic Density}

To determine the atomic density $n$, we utilize the definition of the absorption coefficient $\kappa_{ge}(\omega_L)$, which explicitly accounts for the hyperfine structure and the degeneracy of the ground state:

\begin{equation}
    \kappa_{ge}(\omega_L) = \frac{\pi \omega_{ge} n}{\epsilon_0 \hbar c} \frac{S_{ge}}{2(2I + 1)} g_{\text{D}}(\omega_L, \omega_{ge})
\end{equation}

where:
\begin{itemize}
    \item $2(2I+1) = 16$ is the total degeneracy of the Cesium ground state (nuclear spin: $I=7/2$).
    \item $S_{ge}$ is the line strength for a specific hyperfine transition $|F_g\rangle \rightarrow |F_e\rangle$.
    \item $\omega_{ge}$ is the resonant angular frequency of the transition.
    \item $n$ is the atomic volume density.
    \item $g(\omega_L, \omega_{ge})$ is the normalized lineshape function.
\end{itemize}

The line strength is related to the reduced matrix element by the coefficients $c_F^2$ given in Table \ref{tab:transition_strengths}:
\begin{equation}
    S_{ge} = c_F^2 |\langle L_g || e\vec{r} || L_e \rangle|^2
\end{equation}

\begin{table}[h]
    \centering
    \renewcommand{\arraystretch}{1.5} % Increases vertical row spacing
    \setlength{\tabcolsep}{18pt}      % Increases horizontal column spacing (default is 6pt)
    
    \begin{tabular}{lc}
        \hline
        \textbf{Transition} & $\mathbf{c_F^2}$ \\
        \hline
        $|F_g = 4\rangle \rightarrow |F_e = 3\rangle$ & $7/12$ \\
        $|F_g = 4\rangle \rightarrow |F_e = 4\rangle$ & $5/12$ \\
        $|F_g = 3\rangle \rightarrow |F_e = 3\rangle$ & $7/36$ \\
        $|F_g = 3\rangle \rightarrow |F_e = 4\rangle$ & $7/12$ \\
        \hline
    \end{tabular}
    \caption{Transition strength factors  for the Cs D$_1$ line.}
    \label{tab:transition_strengths}
\end{table}

The square of the reduced matrix element is derived from the natural linewidth $\Gamma$ \cite{steck}:

\begin{equation}
    \langle L_g || e\vec{r} || L_e \rangle = \sqrt{3} \sqrt{\frac{3 \pi \epsilon_0 \hbar c^3 \Gamma}{\omega_{ge}^3}} \quad \Rightarrow \quad |\langle L_g || e\vec{r} || L_e \rangle|^2 = \frac{9 \pi \epsilon_0 \hbar c^3 \Gamma}{\omega_{ge}^3}
    \label{eq:reduced_matrix_element}
\end{equation}

\paragraph{Derivation of the Density Formula}


The experimental signal covers the entire D1 manifold. Therefore, we integrate the absorption coefficient over the full frequency range. The integral of the normalized lineshape function is unity ($\int g_{\text{D}} d\omega = 1$).
\begin{equation}
    \int_{-\infty}^{+\infty} \kappa_{\text{total}}(\omega) d\omega = \frac{\pi \omega_{ge} n}{\epsilon_0 \hbar c} \frac{|\langle L_g || e\vec{r} || L_e \rangle|^2}{16} \sum_{F,F'} c_F^2
\end{equation}

Experimentally, this integral is related to the total area of the fitted Gaussian peaks $\mathcal{A}_{total}$. Converting the frequency units ($d\omega = 2\pi d\nu$) gives the relation $\int \kappa d\omega = \frac{2\pi \mathcal{A}_{total}}{L}$.

Equating the theoretical and experimental integrals:
\begin{equation}
    \frac{2\pi \mathcal{A}_{total}}{L} = \left( \frac{\pi \omega_{ge} n}{\epsilon_0 \hbar c} \right) \left( \frac{|\langle L_g || e\vec{r} || L_e \rangle|^2}{16} \right) \sum c_F^2
\end{equation}

Rearranging to isolate the density $n$:
\begin{equation}
    n = \left( \frac{2\pi \mathcal{A}_{total}}{L} \right) \left( \frac{\epsilon_0 \hbar c}{\pi \omega_{ge}} \right) \left( \frac{16}{|\langle L_g || e\vec{r} || L_e \rangle|^2 \sum c_F^2} \right)
\end{equation}


Substituting everything we obtain a final expression for the atomic density that depends only on the cell length, the measured integrated area, and the fundamental transition constants ($\omega_{ge}, \Gamma$):

\begin{equation}
    n = \frac{2 \omega_{ge}^2 \mathcal{A}_{total}}{\pi L c^2 \Gamma}
\end{equation}

Using this simplified formula with our experimental values ($\mathcal{A}_{total} \approx 1.231132 \times 10^8 \, \text{Hz}$, $\Gamma =  28.743 \, \text{MHz}$, $\omega_{ge} = 2\pi \approx 335.116 \, \text{THz}$) yields the final result of $n \approx 1.345 \times 10^{16} \, \text{m}^{-3}$.
