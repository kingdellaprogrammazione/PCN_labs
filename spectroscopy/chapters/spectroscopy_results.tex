\subsubsection{Experimental Results}

\paragraph{Integrated Absorbance}
To determine the total atomic density, we first calculated the integrated absorbance area ($\mathcal{A}_{total}$) by summing the areas of the four fitted Gaussian peaks. 
The contributions from the four observed hyperfine transitions are:
\begin{itemize}
    \item \textbf{Peak 1:} Area $\approx 3.8303 \times 10^7 \, \text{Hz}$
    \item \textbf{Peak 2:} Area $\approx 1.5576 \times 10^7 \, \text{Hz}$
    \item \textbf{Peak 3:} Area $\approx 2.6074 \times 10^7 \, \text{Hz}$
    \item \textbf{Peak 4:} Area $\approx 4.1681 \times 10^7 \, \text{Hz}$
\end{itemize}

Summing these contributions yields the total integrated absorbance:
\begin{equation}
    \mathcal{A}_{total} = \sum_{i=1}^{4} (A_i \sigma_i \sqrt{2\pi}) \approx 1.21634 \times 10^8 \, \text{Hz}
\end{equation}

\paragraph{Calculation}
Using the values from of the Cesium D1 line \cite{Steck_Quantum_Optics}:
\begin{itemize}
    \item Frequency $\omega_{ge} = 2\pi \times \SI{335.116 048 807(41)}{\tera\hertz}$.
    \item natural linewidth $\Gamma = \SI{28.743}{\mega\hertz}$
\end{itemize}

The final result is:

\begin{equation}
    n \approx 1.33 (27) \times 10^{16} \, \text{m}^{-3}
\end{equation}

\subsubsection{Theoretical Density Calculation}

To provide a benchmark for our experimental results, we calculated the theoretical atomic density of the Cesium vapor at the measured laboratory temperature of $T = \SI{23}{\celsius}$ (\SI{296.15}{\kelvin}).

According to the vapor pressure data for solid Cesium \cite{Steck_Quantum_Optics}, the vapor pressure $P_v$ at this temperature is approximately:
\begin{equation}
    P_v \approx 1.225 (45) \times 10^{-6} \, \text{Torr} \approx 1.633 (60) \times 10^{-4} \, \text{Pa}
\end{equation}



The theoretical density $n_{th}$ is determined using the ideal gas law equation of state $P = n k_B T$, where $k_B$ is the Boltzmann constant:

\begin{equation}
    n_{th} = \frac{P_v}{k_B T} = \frac{1.63 \times 10^{-4}}{(1.38 \times 10^{-23}) (296.15)}
\end{equation}

This yields a theoretical atomic density of:
\begin{equation}
    n_{th} \approx 3.99 (52) \times 10^{16} \, \text{m}^{-3}
\end{equation}



\paragraph{Comparison}
Our experimental density ($n_{exp} \approx 1.33 (27) \times 10^{16} \, \text{m}^{-3}$) is approximately a factor of $3$ lower than this theoretical prediction.
