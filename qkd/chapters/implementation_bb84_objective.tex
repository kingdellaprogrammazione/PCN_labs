This part of the experience has the aim of showing the conceptual and physical realization 
of a system able to execute the BB84 protocol, with some caveats. Indeed what we are showing 
is restricted to be a simple demonstration where we are not producing and measuring 
single photons, but much bigger light signals, due to the complexity and cost of the 
required setup. Nevertheless, the communication system is real, and similar to the real one, 
with lasers, glass fibers, attenuators, beam splitters and half-wave plates. The difference 
instead lies in the fact that the detectors we employed are designed to detect macroscopic signals 
and are calibrated to simulated the behaviour of quantistic detectors, that 
means detectors that click in a 
deterministic way if reached by a sufficiently high light intensity, while they do it
randomly if reached by comparable ones. That simulates the behaviour of a single photon 
wavefunction that can be distributed equally in two paths or concentrated in a single one 
while getting through a beam splitter. 
Another important objective is the understanding of the initial calibration process 
on an optical table, regarding the orientation of the optical instruments, and 
the key sinthesis procedure, with the independent half-wave plate rotations and the sifting step at the end.