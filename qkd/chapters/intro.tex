The techniques that now go under the name of \textit{encryption} have been present for a long time: the oldest attempt 
known where someone tried to protect information is related to Mesopotamian clay tablets, 
approximately around 1500 B.C. 

Since then, mankind never abandoned these tools, but instead 
tried more and more to refine them to allow for exclusively private communication. 

Until 40 years ago (and for more than 3000 years), a common characteristic of encryption techniques was that 
all the ways one could use to hide information were performed with the 
help of only classical information. 
That means that all the procedures exploited, like for example pen's ink or electromagnetic 
waves, follow what we call now classical physics.
People that wanted 
to share messages between themselves had only the options to 
handle macroscopic objects (that could carry information), 
so objects that had all of their properties completely specified at all times, 
that were pretty resistant to environmental interactions and that followed strictly 
the locality principle.

With the understanding of the microscopic world's quantum nature, a new type of 
information emerged: a quantum one. This means that we are now able to encode 
classical information in quantum states, that can be handled in specific ways exclusive 
to quantum, both for what concerned the evolution of the information and its following 
extraction. 

In this chapter we try to enact the first scheme that exploited the new 
peculiarities of quantum information to allow for eavesdropper detection during the synthesis of a symmetric encryption key: BB84 \cite{Bennett_2014}.
This is a protocol that, in its original for, is based on 4 quantum states belonging to a 2-dimensional system: 2 of them are chosen in a way that their 
inner product is equal to $ \sfrac{1}{\sqrt{2}}$, and the other two are their orthogonals. 
A possible physical candidate for this protocol is the photon: it can move at the highest speed available both in free 
air and in fibers, that moreover are already deployed, it can encode 
quantum information in its polarization, a quantum property that lives in a 2 dimensional space (there are also other encoding possible). 
The most important drawback is related to the need for single photons, and so 
single photon sources, single photon detectors, and the attenuation that cannot be 
compensated in classical ways by amplifiers because of the no-cloning theorem.

TODO: describe the protocol better and its components, maybe with the scheme.

\begin{comment}
In this section, students should briefly introduce the general context of quantum communication and QKD.
They should then delve into the description of the BB84 protocol with polarization coding. Finally, they
should list and briefly describe the essential elements of this protocol (sources, detectors, and polarization
optics).
\end{comment}